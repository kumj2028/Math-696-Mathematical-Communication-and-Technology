%%%%%%%%%%%%%%%%%%%%%%%%%%%%%%%%%%%%%%%%%%%%%%%%%%%%%%%%%%%%%%%
%
% Welcome to writeLaTeX --- just edit your LaTeX on the left,
% and we'll compile it for you on the right. If you give
% someone the link to this page, they can edit at the same
% time. See the help menu above for more info. Enjoy!
%
%%%%%%%%%%%%%%%%%%%%%%%%%%%%%%%%%%%%%%%%%%%%%%%%%%%%%%%%%%%%%%%

% --------------------------------------------------------------
% This is all preamble stuff that you don't have to worry about.
% Head down to where it says "Start here"
% --------------------------------------------------------------
 
\documentclass[12pt]{article}
 
\usepackage[margin=1in]{geometry}
\usepackage{amsmath,amsthm,amssymb}
\usepackage{enumitem}

\setlist[enumerate,1]{label={(\alph*)}} %this changes enumerate to (a),(b),...

\usepackage{graphicx} %package to manage images

\newcommand{\A}{{\mathcal{A}}}
\newcommand{\C}{{\mathbb C}}
\newcommand{\CC}{{\mathcal{C}}}
\newcommand{\N}{{\mathbb N}}
\newcommand{\R}{{\mathbb R}}
\newcommand{\Q}{{\mathbb Q}}
\newcommand{\Z}{{\mathbb Z}}

\newcommand{\Aut}{{\rm Aut}}
\newcommand{\End}{{\rm End}}
\newcommand{\Hom}{{\rm Hom}}
\newcommand{\id}{{\rm id}}
\newcommand{\Ima}{{\rm Im}}
\newcommand{\Ker}{{\rm Ker}}
\newcommand{\Mor}{{\rm Mor}}
\newcommand{\Rad}{{\rm Rad}}

\renewcommand\labelitemi{-} %this changes itemize bullet points to dashes (-)

\usepackage{listings}
\usepackage{xcolor}

%New colors defined below
\definecolor{codegreen}{rgb}{0,0.6,0}
\definecolor{codegray}{rgb}{0.5,0.5,0.5}
\definecolor{codepurple}{rgb}{0.58,0,0.82}
\definecolor{backcolour}{rgb}{0.95,0.95,0.92}

%Code listing style named "mystyle"
\lstdefinestyle{mystyle}{
  backgroundcolor=\color{backcolour}, commentstyle=\color{codegreen},
  keywordstyle=\color{magenta},
  numberstyle=\tiny\color{codegray},
  stringstyle=\color{codepurple},
  basicstyle=\ttfamily\footnotesize,
  breakatwhitespace=false,         
  breaklines=true,                 
  captionpos=b,                    
  keepspaces=true,                 
  numbers=left,                    
  numbersep=5pt,                  
  showspaces=false,                
  showstringspaces=false,
  showtabs=false,                  
  tabsize=2
}

%"mystyle" code listing set
\lstset{style=mystyle}
 
\newenvironment{theorem}[2][Theorem]{\begin{trivlist}
\item[\hskip \labelsep {\bfseries #1}\hskip \labelsep {\bfseries #2.}]}
{\end{trivlist}}
\newenvironment{lemma}[2][Lemma]{\begin{trivlist}
\item[\hskip \labelsep {\bfseries #1}\hskip \labelsep {\bfseries #2.}]}
{\end{trivlist}}
\newenvironment{exercise}[2][Exercise]{\begin{trivlist}
\item[\hskip \labelsep {\bfseries #1}\hskip \labelsep {\bfseries #2.}]}
{\end{trivlist}}
\newenvironment{problem}[2][Problem]{\begin{trivlist}
\item[\hskip \labelsep {\bfseries #1}\hskip \labelsep {\bfseries #2.}]}
{\end{trivlist}}
\newenvironment{question}[2][Question]{\begin{trivlist}
\item[\hskip \labelsep {\bfseries #1}\hskip \labelsep {\bfseries #2.}]}
{\end{trivlist}}
\newenvironment{corollary}[2][Corollary]{\begin{trivlist}
\item[\hskip \labelsep {\bfseries #1}\hskip \labelsep {\bfseries #2.}]}
{\end{trivlist}}

\newenvironment{solution}{\begin{proof}[Solution]}{\end{proof}}
 
\begin{document}
 
% --------------------------------------------------------------
%                         Start here
% --------------------------------------------------------------
 
\title{Project 3 Writing An Assessment
\\ Week 5: The Product and Quotient Rules and Derivatives of Trig Functions}
%replace X with the appropriate number
\author{Mengxiang Jiang\\ %replace with your name
Math 696 Mathematical Communication and Technology} %if necessary, replace with your course title
 
\maketitle
 
\begin{problem}{1} %You can use theorem, exercise, problem, or question here.
Let
\[
  f(x) = (3x^2 - 2x)\sin x.
\]
Compute $f'(x)$.

\begin{enumerate}
  \item $(6x - 2)\cos x +(3x^2 - 2x)\sin x$
  \item $(6x - 2)\sin x +(3x^2 - 2x)\cos x$
  \item $(6x - 2)\sin x -(3x^2 - 2x)\cos x$
  \item $(6x - 2)\cos x -(3x^2 - 2x)\sin x$
\end{enumerate}
\end{problem}

\begin{problem}{2}
Let
\[
  g(x) = \frac{\tan x}{x}.
\]
Compute $g'(x)$.

\begin{enumerate}
  \item $\displaystyle \frac{x\sec^2 x -\tan x}{x^2}$
  \item $\displaystyle \frac{\sec^2 x -\tan x}{x}$
  \item $\displaystyle \frac{x\sec^2 x +\tan x}{x^2}$
  \item $\displaystyle \frac{x\tan^2 x -\sec^2 x}{x^2}$
\end{enumerate}
\end{problem}

\pagebreak

\begin{problem}{3} %You can use theorem, exercise, problem, or question here.
(Note: One thing I didn't like about this week's lecture was that\\
$\lim_{x \to 0} \frac{\sin x}{x} = 1$ was given without proof. I think it is 
important to prove this before it is used in the proof of the derivative of 
$\sin x$. Here is an alternative way to prove the derivative of $\sin x$ without
using this limit.)\\
One of the most famous identities in mathematics is Euler's formula:
\[
  e^{ix} = \cos x + i\sin x.
\]
This identity also gives us a way to represent the sine and cosine functions
as exponential functions by taking advantage of the even and odd properties of 
the sine and cosine, with $e^{-ix} = \cos(-x) + i\sin(-x) = \cos x - i\sin x$
and adding or subtracting the two equations:
\[
  \sin x = \frac{e^{ix} - e^{-ix}}{2i}, \quad
  \cos x = \frac{e^{ix} + e^{-ix}}{2}.
\]
Using this representation as well as $\frac{d}{dx}e^{cx} = ce^{cx}$, 
compute the derivative of $\sin x$.\\
(The derivative of the exponential function was covered in the previous week.)
\end{problem}

\pagebreak

\begin{problem}{1 Solution}
To compute $f'(x)$, we apply the product rule as well as $(\sin x)' = \cos x$:
\[
  f'(x) = (3x^2 - 2x)'\sin x + (3x^2 - 2x)(\sin x)' 
  = (6x - 2)\sin x + (3x^2 - 2x)\cos x.
\]
This is choice (b).\\
This problem is worth 3 points and can be given partial credit as follows:
\begin{itemize}
  \item 1 point for either writing the product rule or applying it correctly.
  \item 1 point for computing the derivative of $3x^2 - 2x$ correctly.
  \item 1 point for computing the derivative of $\sin x$ correctly.
\end{itemize}
\end{problem}

\begin{problem}{2 Solution}
To compute $g'(x)$, we apply the quotient rule as well as $(\tan x)' = \sec^2 x$:
\[
  g'(x) = \frac{(\tan x)'\cdot x - \tan x\cdot (x)'}{x^2}
  = \frac{x\sec^2 x - \tan x}{x^2}.
\]
This is choice (a).\\
This problem is worth 3 points and can be given partial credit as follows:
\begin{itemize}
  \item 1 point for either writing the quotient rule or applying it correctly.
  \item 1 point for computing the derivative of $\tan x$ correctly.
  \item 1 point for computing the derivative of $x$ correctly.
\end{itemize}
\end{problem}

\begin{problem}{3 Solution}
Use the identity:
\[
  \sin(x) = \frac{e^{ix} - e^{-ix}}{2i}
\]
Differentiate both sides with respect to $x$:
\[
\begin{aligned}
  \frac{d}{dx}\sin(x) &= \frac{d}{dx}\left(\frac{e^{ix} - e^{-ix}}{2i}\right)\\
  &= \frac{1}{2i} \left( \frac{d}{dx} e^{ix} - \frac{d}{dx} e^{-ix} \right)\\
  &= \frac{1}{2i} \left( i e^{ix} + i e^{-ix} \right)
\end{aligned}
\]
Factor out $i$ and simplify:
\[
\begin{aligned}
  \frac{d}{dx}\sin(x) &= \frac{i}{2i} (e^{ix} + e^{-ix})\\
  &= \frac{e^{ix} + e^{-ix}}{2} = \cos(x).
\end{aligned}
\]
This problem is worth 3 points and can be given partial credit as follows:
\begin{itemize}
  \item 1 point for differentiating the exponential functions.
  \item 1 point for proper algebraic manipulations.
  \item 1 point for showing that the final answer is the identity for $\cos x$.
\end{itemize}
\end{problem}

% --------------------------------------------------------------
%     You don't have to mess with anything below this line.
% --------------------------------------------------------------
\end{document}